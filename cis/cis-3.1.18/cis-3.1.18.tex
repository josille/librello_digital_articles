\documentclass[10pt,a4paper]{article}
\usepackage[utf8]{inputenc}
\usepackage[english]{babel}
\usepackage[square, numbers, sort&compress]{natbib}
\usepackage{graphicx}
\usepackage{float}
\usepackage{amsmath}
\usepackage{amsfonts}
\usepackage{amssymb}
\usepackage{media9}
\usepackage{color}

\definecolor{librelloCOLOR}{RGB}{0,85,0}		% Green
%\definecolor{librelloCOLOR}{RGB}{128,0,0} 	% Red

\usepackage{fancyhdr}
\usepackage{lastpage}	
\usepackage{parskip}
\usepackage[scaled]{helvet}
\usepackage{blindtext}
\usepackage{sectsty}
\usepackage{multicol}
%\usepackage[svgnames]{xcolor}
\usepackage[labelfont={color=librelloCOLOR,bf}, labelsep=period]{caption}
\renewcommand*{\familydefault}{\sfdefault}
\usepackage[left=1.75cm,right=1.75cm,top=1.75cm,bottom=3.75cm]{geometry}
\usepackage{titlesec}
\usepackage{svg}
\usepackage{flushend}
\PassOptionsToPackage{normalem}{ulem}
\usepackage{ulem}
	\providecolor{added}{rgb}{0,0,1}
	\providecolor{deleted}{rgb}{1,0,0}
	%% Change tracking with ulem
	\newcommand{\added}[1]{{\color{added}{}#1}}
	\newcommand{\deleted}[1]{{\color{deleted}\sout{#1}}}
\usepackage{setspace}
\usepackage[hyphens]{url}
\usepackage[hidelinks]{hyperref}
\urlstyle{same}
\usepackage{soul}
\usepackage{enumitem} 
\raggedcolumns
\flushcolumns
\usepackage{etoolbox}
\usepackage{caption}
\newcommand{\cfbox}[2]{%
    \colorlet{currentcolor}{.}%
    {\color{#1}%
    \fbox{\color{currentcolor}#2}}%
}

\titleformat{\section}
{\color{librelloCOLOR}\normalfont\bfseries\filright}
{\color{librelloCOLOR}\thesection.}{0.5em}{}

\titleformat{\subsection}
{\color{librelloCOLOR}\normalfont\itshape\filright}
{\color{librelloCOLOR}\thesubsection.}{0.5em}{}

\usepackage{array} %criando coluna de largura fixa alinhada a esquerda
\newcolumntype{L}[1]{>{\raggedright\let\newline\\\arraybackslash\hspace{0pt}}p{#1}}
\renewcommand{\arraystretch}{1.3}
\titlespacing\section{0pt}{12pt}{12pt}
\titlespacing\subsection{0pt}{12pt}{12pt}
\titlespacing\subsubsection{0pt}{12pt}{12pt}	
\renewcommand*{\refname}{References and Notes}

\fancypagestyle{document}{
	\renewcommand{\footrulewidth}{0pt}
	\renewcommand{\headrulewidth}{0pt}
	\renewcommand{\footrulewidth}{0pt}
	\renewcommand{\headrulewidth}{0pt}
	\renewcommand{\footskip}{40pt}
	\cfoot{\normalfont\thepage}
	\rhead{}\lhead{}
}

\fancypagestyle{firstpage}{
	\renewcommand{\footrulewidth}{0pt}
	\renewcommand{\headrulewidth}{0pt}
	\renewcommand{\footrulewidth}{0pt}
	\renewcommand{\headrulewidth}{0pt}
	\renewcommand{\footskip}{70pt}
	\lhead{Challenges in Sustainability $\mid$ 2015 $\mid$ Volume 3 $\mid$ Issue 1 $\mid$ Pages \thepage--\pageref{LastPage} \\DOI: 10.12924/cis2015.03010018\\ ISSN: 2297--6477}
	\rhead{\includegraphics[height=0.59in]{CiS.eps}}
	\lfoot{\footnotesize © 2015 by the authors; licensee Librello, Switzerland. This open access article was published\\ under a Creative Commons Attribution License (\url{http://creativecommons.org/licenses/by/4.0/}).}
	\cfoot{}
	\rfoot{\vspace*{-24pt}\includegraphics[height=0.49in]{librello.eps}}

}

\makeatletter
\def\NAT@def@citea{\def\@citea{\NAT@separator}}
\makeatother

\begin{document}
\flushcolumns
\raggedcolumns



\pagestyle{document}
\thispagestyle{firstpage}


%\vspace*{70pt}
\vspace*{50pt}
\setlength{\parindent}{0cm}


\textit{Book Review}
\vspace*{-12pt}

\begin{center}
\line(1,0){500}
\end{center}

\vspace*{6pt}
\begin{flushleft}
\begin{LARGE}
\textbf{{\color{librelloCOLOR} A Review of ``The Politics of Sustainability: Philosophical Perspectives''}}\\
\end{LARGE}
\vspace*{6pt}
Published: 4 December 2015\\
\vspace*{6pt}
\textbf{{\color{librelloCOLOR}Keywords:}} ethics; intergenerational responsibility; philosophical perspectives; sustainability; technological leapfrogging
\end{flushleft}

\vspace*{-18pt}
\begin{center}
\line(1,0){500}
\end{center}
\setcounter{page}{18}

%\vspace*{10mm}
\vspace*{12mm}

\begin{multicols}{2}

\cfbox{librelloCOLOR}{
\begin{minipage}{8.1cm}
\vspace{\baselineskip}
{\color{librelloCOLOR}\textbf{\noindent The Politics of Sustainability: Philosophical Perspectives}\\ Birnbacher D, Thorseth M (Eds.)\\
Routledge: London, UK. 2015\\
234 p.; ISBN: 978-1138854291}
\vspace{10pt}
\end{minipage}
}

\vspace{\baselineskip}
\vspace{\baselineskip}

\setlength{\parindent}{0.5cm}
\setlength{\parskip}{0cm}
\setlength{\bibsep}{0cm}


\noindent Concerns about sustainable development are not a recent phenomenon. Societal problem-solving efforts within this realm have focused on concrete problems such as the preservation of fisheries, forests and national reserves. `The Politics of Sustainability' has been discussed extensively in  literature, particularly after the publication of the Brundtland Commission's `Our Common Future' report in 1987 \citep{r1} emphasizing inter-generational responsibilities involving economic, environmental and social aspects. Among other areas, the authors of the report highlighted the challenge of global climate change resulting from, amongst other things, unsustainable patterns of consumption. `The Politics of Sustainability: Philosophical Perspectives', edited by Dieter Birnhacher and May Thorseth, brings  a new angle into the discussion of the politics of sustainable development: ethical considerations.  

In Part 1, contributors to the book discuss---in a somewhat pessimistic tone---the determinants of non-sustainable behavior, which are lack of motivation; institutions stressing and the difficulties  the democratic governments face when  implementing the actions needed to protect future generations. The authors also underscore  the challenges in pressuring politicians to make real progress through sustainable policies. This is particularly difficult given the existence of more immediate short-range challenges such as economic crises, high unemployment, and maintaining an upstanding national position in the international political rat-race.

In Part 2 of the book, the contributors present and discuss the strong moral and philosophical dimensions of policy implementation, setting them up, in Part 3, for the proposition of establishing a new fourth institution of government beyond the legislative, executive and judicial. This branch would be responsible to ensure ``that the interest of future generations be taken into account within today's decision-making-process''. The proposal is indeed original. Its adoption in democratic countries will face the same difficulties as pointed out in Part 1: to implement practical actions to reduce greenhouse gas emissions and to protect future generations.

It is my view, that, governance is more complicated than is portrayes in the book and is not completely compatible with different democratic systems around the globe. To state that ``the incentives for politicians in a democratic system are in maintaining power and securing re-election'' (p. 55) is an over simplification. History has demonstrated that there have been many occasions in which even problematic politicians or legislative bodies of governments have risen to the task of confronting great societal challenges. An example is the decision in the United States Congress to support President Roosevelt in joining World War II, despite strong societal sentiments for isolationism at the time. The existence of great leaders is essential for democracies to move and history has demonstrated that they frequently appear at critical junctions. This is why there is optimism with societies in making continuous strides in sustainable development, including the more difficult questions such as the elimination of poverty and averting catastrophic changes in the climate.

Technology is one powerful instrument. As technical solutions improve, they become less expensive. This creates opportunities for developing countries and means they do not have to retrace the steps followed by industrialized countries; they can “leapfrog” over many of the developmental steps and avoid many of the problems caused by industrialization. An example is that of Brazil’s response to the problem of greenhouse gas emissions (GHG) from the use of gasoline in automobiles.  As an alternative, ethanol produced from sugarcane (a renewable crop) was used to replace a large fraction of the gasoline, thus reducing the GHG emissions in the country by approximately 10\% \citep{r2}.

Ethical considerations, of course, are of great importance  when democracies move in more sustainable directions.  We must not forget, however, that the existence of technical solutions can also help governments to implement sustainable solutions at an expeditious pace. This book will, undoubtedly, be of great interest to people less interested in economics and environment, and/but fundamental to the philosophical perspectives of sustainable development.

\setlength{\parindent}{0cm}
\rule{\columnwidth \color{librelloCOLOR}}{1pt}

José Goldemberg

Electrotechnical and Energy Institute (IEE), University of São Paulo (USP), Brazil ; E-Mail: goldemb@iee.usp.br

\rule{\columnwidth \color{librelloCOLOR}}{1pt}
\vspace{18pt}
\end{multicols}

\begin{multicols}{2}
\bibliographystyle{vancouver}
\bibliography{240Library.bib}
\end{multicols}


\end{document}
